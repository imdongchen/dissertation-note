\section{Collaboration and Awareness}

While several attempts, as discussed above, have been made to implement support for various structured techniques, most of them are designed for individual use only. However, prior work \citep{Chin2009, Warner2008} emphasized that the work of information analysis at non-trivial scales is fundamentally collaborative, and that the single person operating the software is a bottleneck to the team process. A report from National Intelligence \citep{Vision2015} states that intelligence analysis increasingly becomes a collaborative enterprise, with the focus shifting “from traditional emphasis on self-reliance toward more collaborative operations ”. This is a major change from the traditional concept of intelligence analysis as largely an individual activity.

Structured techniques are reported to be best utilized in a group effort so that collaborative opinions insights could be built \citep{Heuer1999}. The transparency of structured techniques ensures that divergent or conflicting perspectives among analysts are heard and seriously considered early in the analytic process. For example, ACH is often used in team collaboration because ACH leaves an audit trail showing what hypotheses were identified, what evidence was used, and how they were interpreted. If teammates disagree, the ACH matrix could highlight the source of disagreement, and be used to ground further team discussion. Te@mACH is a recent advance in ACH technology, developed by Globalitica, LLC, in 2010. It is designed as a collaborative version of PARC ACH. It allows distributed analysts to work on the same ACH matrix simultaneously. Other tools for supporting structured techniques, however, mostly remain a single-user application. 


With single-user applications, teams of analysts have to rely on external view sharing tools \citep{Greenberg1990} or manually share notebooks and screenshot graphs. This has the consequence that the tools are employed only at specific points in an analysis, often only in the early stages during which analysts are working on their own. Indeed, the use of these tools directly diminishes a team’s activity awareness, requiring repeated manual resynchronizing to identify redundant or missing pieces of information, analyzes of information, and analytic hypotheses. This can be mitigated through maintaining and pooling individual analysis reports using Google’s real-time editing tool set, Google Doc, Google Sheets, Google Calendar, etc. These tools enable collaborative document editing, and provide a revision history list of changes made to a document by each collaborator, supporting awareness of teammates’ contributions. However, the Google tools themselves are not integrated; each tool manages its own data representation. Thus an event report in Google Doc cannot be exported to Google Calendar or Google Map. Non-existent or weak support for coordinating multiple visual representations could undermine activity awareness. We found this to be a specific source of problems in our study of spontaneously created physical visualizations in the information analysis task \citep{Carroll2013}.

Vision 2015 also notes a shift “away from coordination of draft products toward regular discussion of data and hypotheses early in the research phase”. This emphasizes the importance of close coupling collaboration in the analytic process, as opposed to simple coordination as the final step. A key enabler of such effective and close collaboration is activity awareness, a critical concept for research in computer-supported collaborative work (CSCW). In prior work Carroll and his team \citep{Carroll2003, Carroll2011j} developed a conceptual framework, conducted design research investigations of software tools \cite{Carroll2003,Carroll2009i,Ganoe2003}, carried out field studies \citep{Ganoe2003,schafer2008emergency}, and made controlled experimental studies \citep{Convertino2008,Convertino2009,Convertino2011}. That body of work produced several results: Activity awareness can be measured, though this entails methods beyond those traditionally employed in awareness research, including conversation analysis \citep{sacks1995lectures} and artifact analysis \citep{fleming1974artifact,pearce1994thinking}. Activity awareness grows with the development of reciprocal knowledge about, experience with, and trust of partners. It is facilitated when the sharing of collaborator-specific knowledge is implemented as an explicit, public and documented interaction within the collaborative activity; indeed, activity awareness can be increased beyond measured levels in face-to-face interactions \citep{Convertino2011}, a rare result in computer-supported cooperative work (CSCW) sometimes referred to as “beyond being there” \citep{hollan1992beyond}. The growth of activity awareness qualitatively changes coordination strategies that partners adopt (for example, early in a collaboration, members pull information from partners by voicing their needs and concerns to the team, but subsequently, as team members become more aware of the responsibilities and roles of other members, they selectively push information to fellow members). Thus, becoming a team does not merely speed up or improve coordination; it changes what coordination means and what members do to coordinate. Activity awareness engages active and meta-cognitive strategies for improving team cognition, for example, through negotiating and sharing ad hoc representational artifacts \citep{Carroll2013}. Finally, activity awareness can be effectively supported with interactive tools, and the benefits of supporting it can be measured. 

More nuanced team behavior is reported in empirical studies. For example, \cite{Chin2009}'s study observed that the analysts  preferred to review the original source material.  Analysts had distinctive and strong beliefs about the artifacts they created themselves, yet claimed they would not trust another analyst’s artifacts, although they believed they could achieve better results by collaborating with others. In another study, \cite{Kang2012b} found that analysts worked as a team throughout the process and the degree of collaboration differs depending on the type of task. These findings reveal details of team behavior and are useful in understanding and designing for better collaboration features.


