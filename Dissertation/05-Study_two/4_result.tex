\section{Result}

\subsection{Confirming result from study one}

Many feedbacks from students in study two echo and confirm results from study one. For example, users like the feature that maps out annotated data objects to different visualizations. This allows them to put various perspectives into the otherwise unstructured text documents.

\begin{quote}
\emph{We feel that this service was helpful in that we could visually see the relationships that we mapped out as well as add our own parts to it without needing to email them to each other or having multiple networks.(P178)}
\end{quote}

Users also mentioned the usage of the new timeline. They appreciate the capability to match individual events against critical events; thus they can quickly target data of interest. The multiple tracks in the timeline view also encourage them to use the filter feature. 

\begin{quote}
\emph{It helped because we were able to see who was where at what times so when the crimes were taking place we knew who was near the crime or who was busy etc.(P150)}
\end{quote}

\begin{quote}
\emph{The main tool used to help eliminate subjects was by using the filter tool to see what suspects were not at the location of the thefts at the times for each incident (P188)}
\end{quote}

Users rate highly of the collaboration features. They feel the idea of sharing one common analytic artifact changes the way they collaborate. 

\begin{quote}
\emph{This differs because I can actually see what my team is doing as they are doing it, but we can also alter the information and that would change the entire network. Changing the network and altering information also updated and altered my own information, which was something new compared to the other tools I have used}
\end{quote}

One user also comments that he would first check his teammates' update before he sets out for his own analysis. 

\begin{quote}
\emph{Whenever I would work, I would go to the network first to see the connections that had been made. (P186)}
\end{quote}

Similar to the result in study one, students frequently associate CAnalytics with Google Docs and think of it as the Google Doc for analysts. 

\begin{quote}
\emph{CAnalytics tool is quite helpful when compared with other information analysis platforms such as google docs because it helps in real time annotating of various persons, relationships, events, etc. It also is a good tool when multiple people need to work on the same document and since it provides updates it helps reduce redundant work. Also, it provides additional features such as Timeline, Map, Network to help analyze information. (P193)}
\end{quote}

\begin{quote}
\emph{It gave us the ability to create and see the visuals that we would have only been writing about in programs like Google Docs. (P178)}
\end{quote}


\subsection{Feedback on hypothesis development tool}

Feedback on the structured approach for hypothesis development is mixed. We observe two major ways teams use this feature. First, they use it as a space for insight curation. As individuals go through annotation and visualization, they create notes about their hypotheses, findings, or ideas.

\begin{quote}
	\emph{The hypothesis was nice in the sense that it was a clean space to put final thoughts (p100)}
\end{quote} 

\begin{quote}
	\emph{As we went throughout our work we noted any important information and relationships that we found in the raw data to create hypotheses (P183)}
\end{quote}

These ideas then become a resource to which they refer when they compose their final report. This can be a useful feature as it helps analysts to focus on high level hypotheses instead of relooking at underlying data and visualizations. 

\begin{quote}
	\emph{I read through the hypothesis and used that to form my answers to the questions. (p183)}
\end{quote}


Secondly, participants use it to narrow down analysis and direct the team to focus on analyzing a particular aspect. When one team member creates a hypothesis, partners look into that hypothesis, retrieve the context where that hypothesis was generated, and help validate it connecting with their own knowledge. 

\begin{quote}
	\emph{It helped us focus and narrow in on an event and suspect.(P191)}
\end{quote}
\begin{quote}
	\emph{I confirmed the results by filtering out the maps, networks, and tables to confirm the conclusion. (P188)}
\end{quote}
\begin{quote}
	\emph{When my teammate shared a hypothesis about where some of the suspects work, I responded by using that hypothesis and applying it to my own analysis to see if I can make a connection between the hypothesis and the theft. (P198)}
\end{quote}


On the other hand, some participants commented that the way a hypothesis is generated does not follow their workflow. They reflect that when they find something interesting during their analysis, they would share it immediately with teammates (verbally). 

\begin{quote}
	\emph{We all agreed with each other when we posted hypotheses because we worked together the whole time (p100)}
\end{quote}

\begin{quote}
	\emph{We wrote it together so we didn’t really respond to it.(p150)}
\end{quote}

The team discusses possible hypotheses and comes up with a most reasonable one together. In other words, the team works closely in a synchronous manner to generate a hypothesis that everyone would agree on. This differs from the model our tool design was based on, which encourages frequent turn-around in developing a hypothesis; each turn-around would be more \textit{asynchronous} between which teammate would spend time assessing the validity of the last hypothesis. This closer approach to generating hypotheses seems prevalent as several teams mention the mismatch between the tool design and their workflow. This result is interesting as it is in contrast with the feedback on data annotation and visualization, for which the tool supports a similar structured technique. Participants mostly like those features as ``they reduced the efforts to frequently keep updated with the team''. We will extend exploration into this contrast in the discussion section.



One student mentions the scalability issue. When the team creates too many hypotheses, it becomes less efficient to continue developing these hypotheses. 

\begin{quote}
	\emph{we enjoyed making hypothesis together, and the tool quickly got out of hand when we all just tried tossing in info (P204)}
\end{quote}

It is true that the tool currently does not include any feature to search or manage hypotheses yet. The display is also not optimized for a large number of hypotheses. The goal of this study is to explore the feasibility of such a structured hypothesis development approach. But it will be an interesting future research topic to investigate how to enable large scale, long term hypothesis development and management. 

A few participants also mention the discoverability and usability issue of the feature. The feature is only accessible under a secondary menu in the system, thus it is not straightforward to start using the feature. The success of the tool also depends on how the whole team is using it. If one made a hypothesis but never got a response from the team, they gave up.

\begin{quote}
\emph{My team did not share a hypothesis, I created a couple but I did not get a response from my team (P186)}
\end{quote}

\begin{quote}
\emph{It was useful to see what they were indicating, but I still had to call them over to fully explain their rationale and logic. It was a good start point, but without them explaining in person to me what patterns they were seeing I was having difficulty making the same conclusions. (P204)}

\end{quote}

Interestingly, one participant did not use the feature but reflected that they probably should. 

\begin{quote}
\emph{We never completed a hypothesis within CAnalytics because it did not occur to us to do so. However, looking back, it might have been helpful to solve the case. (P197)}
\end{quote}

We do not know what makes him/her think that way, but research has observed such behavior before that might account for the reason. A structured technique demands more effort as users must explicitly externalize their thought, which takes more time than verbally communicating with teammates. Externalization, however, in the long run, benefits the team as they have an artifact for team reference. 



