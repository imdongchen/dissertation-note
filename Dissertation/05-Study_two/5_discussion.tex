\section{Discussion}

The design goal of CAnalytics V2 is to support higher-level collaborative sensemaking while confirming the positive effect of integrating low-level data annotation and visualization efforts established in study one. Information analysis can be roughly divided into three key levels \citep{Pirolli2005,Kang2014a}: (1) identifying critical evidence; (2) organizing evidence into a schema; and (3) generating hypotheses based on evidence schema. Accordingly, CAnalytics provides data annotation, data visualization, and hypothesis threading tool for each level. Study one has suggested a positive impact with the integration of data annotation and visualization. Study two aims to provide a linkage between hypothesis development and data visualization and investigates the feasibility in a collaborative setting.

While collaboration occurred throughout the project, we observed that the degree of closeness in which they collaborated varied depending on the phase of analysis. When they set out the project, they usually had a close discussion on how they were going to tackle the problem strategically and how they would divide their work. This is often known as \emph{strategic coordination}, or functional collaboration \citep{Kang2011}. CAnalytics does not have a feature to explicitly support this, but we do observe participants discussing over the messaging tool. 

The team usually then divided up the work and worked on their part individually. They still collaborated by sharing evidence they annotated, but the collaboration was looser because it was mostly \emph{content} sharing without a need for deep discussion. The real time sharing feature provided by CAnalytics helped facilitate collaboration. 

We observed a stronger demand for close collaboration in hypothesis development than in data annotation or visualization. We speculate that collaboration in higher level sensemaking requires a higher volume of information to be exchanged. While data annotation activities share only a piece of annotation, hypothesis development requires the sharing of a rich amount of evidence and schema at a high frequency. And based on the evidence, there can more than one direction analysts can draw from. The team needs a more synchronous, closer channel to exchange their ideas. This is probably why they would prefer to discuss over hypotheses together and compose hypothesis collectively.

This nuance in collaboration closeness suggests us to consider accounting for the different needs in our design for groupware. Teammates in loosely coupled collaboration are weakly dependent on one another and can function autonomously, often without the need for immediate clarification or negotiation with others. Yet closely coupled analysts have patterns of collaboration that distinguish them from other types of teamwork, and groupware systems that are designed to support both couplings must address these differences. 



Another goal of this research is to explore alternatives to existing structured techniques such as Analysis of Competing Hypotheses (ACH). There are several key differences in our approach to hypothesis development.

First, it takes less effort to start generating a hypothesis. ACH asks analysts to cover a wide range of hypotheses in the very beginning; therefore it demands a decent level of experience and knowledge. It takes serious commitment to come up with all possible scenarios and often is intimidating enough to drive away junior analysts. In contrast, our approach is opportunistic. As analysts annotate key evidence in documents, they can create a hypothesis whenever they see a connection between two pieces of evidence. It makes analysts easier to get started and gain momentum to continue.

Second, the hypothesis development process is integrated with evidence annotation and analysis. Analysts can provide an associated visualization state for each hypothesis, and that visualization provides the full context with which partners can validate the hypothesis. Lacking context is often a critique of ACH \citep{Gelder2008} because each evidence/hypothesis is treated on its own. 

Third, hypotheses can be more flexibly structured. The matrix structure in ACH assumes hypotheses are isolated and are on a single level. It asks all hypotheses to be entered individually on the top row of the matrix. In reality, however, hypotheses can be more or less abstract or general. A hypothesis can have sub-hypotheses, and a hypothesis can be built upon another one. The threading structure in our approach allows for a more flexible way to develop hypotheses. A sub-thread can support, refute, or extend the parent thread. 

