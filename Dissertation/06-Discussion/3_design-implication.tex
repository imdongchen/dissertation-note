\section{Education in collaborative information analysis}

An important direction of this research is to explore how such a collaborative instrument could be employed to support collaborative learning in education. We studied an introductory intelligence analysis course, but many other courses involve heavy information analysis tasks, in which snippets of information must be annotated and analyzed with respect to concepts, connections, weights of evidence, and pros and cons of opinions. This is similar to generating and evaluating hypotheses in our case. Thus one could imagine courses in business, new media, and those involving heavy literature review adopting this tool similar to the case described here.

The goal is to shape students' behavior towards more collaborative learning. With traditional single-user tools, students often employ a divide-and-conquer strategy; they divide their job responsibility, work individually on their own part, and put the results together in the end for report submission. In our study, we observed that students spontaneously conducted closer collaboration and enjoyed being able to contribute simultaneously.
% cite education paper in intelligence analysis

In addition to shaping the \textit{learning} behavior of students, a collaborative supporting tool can also change how an instructor \textit{teaches} analytic skills. Anecdotal reflections from the instructor in our classroom study emphasized the role of system support in \textit{instructor intervention}. During our classroom study, the instructor frequently went over to the student's desk and checked how they were doing. The instructor commented that he valued students' analytic process as much as their final report. His emphasis on analytic process is consistent with the value of the intelligence community, who claimed that analysts \textit{``should think about how they make judgments and reach conclusions, not just about the judgments and conclusions themselves''} \citep{Heuer1999}. However, the process is hard or costly to capture for assessment in reality. In the context of education, for example, all students are conducting analysis at the same time. Without explicit support, the instructor has limited supervision over the process.

The collaborative learning instrument is likely to provide an opportunity to get a student's analytic traces preserved and assessed because the system has already captured and saved user interactions. These logs are currently saved for research purposes in our study, but potentially they can be displayed to the instructor for assessment purposes. Taking advantage of the system's synchronization capability, student's behavior can be streamed to a ``monitoring dashboard'' in real time, from which the instructor could check student's progress and take early intervention if any team deviates from the expected path.