
Collaborative information analysis involves modeling and representation of complex information space through synchronous and asynchronous team interactions over extended periods. Effectiveness of collaboration determines team performance, and thus the justice of decisions made or solutions proposed. Effective collaboration is challenging because extra efforts are required to develop and maintain team awareness, in addition to the already high demanding cognition required to analyzing the information itself. 

Numerous tools and techniques have been developed for information analysis, but the majority are designed for individual use rather than for collaborative use. On the other hand, several groupware have been developed to facilitate team communication and coordination, yet they lack support for information modeling and representation. A gap exists in the research of understanding team behaviors in tasks of complicated information analysis and supporting effective collaborative information analysis. Besides, it is challenging to empirically evaluate collaborative analytic tools: complex analytic scenarios are difficult to model in lab studies, and real cases and professional analysts are often limited to access in reality.

Developing and evaluating effective tool support for collaborative information analysis is a multifaceted and complex problem. Researchers must understand the workflow of modeling and analyzing complex information, and develop tools to integrate, rather than interrupt the workflow. Meanwhile, the tools should provide support for collaboration seamlessly, offloading part of analysts' burden required for maintaining team awareness. 

My research starts with a task analysis of an example of collaborative information analysis in the real world: an undergraduate course of intelligence analysis at Pennsylvania State University. I described student analysts' workflow and their team behavior with current tooling. Specifically, the study observes that structured techniques are frequently employed but lack serious collaborative support. Based on the observation, five design objectives are proposed for a better collaborative tool. These objectives drive the design and development of a new tool out from this dissertation, \textit{CAnalytics}, an integrated analytic workspace that supports real-time collaborative data annotation for information modeling and collaborative data visualization for information analysis. 

\textit{CAnalytics} is evaluated in two classroom studies, in which students are being trained to become professional information analysts. I did a quantitative analysis on system logs and student reports, as well as a qualitative analysis of questionnaires. The first study emphasizes assessment of integrating structured information modeling and visualization in a single collaborative workspace. I analyze different team behaviors on multiple dimension, and their interaction with team performance. The second study focuses on supporting a higher-order information analysis activity, i.e. collaborative hypothesis development, using a structured approach. Both studies contribute to the understanding of analysts' team behavior in collaborative information analysis and the role of computing tools in support of both collaboration and information handling.

In summary, this dissertation contributes an understanding of how analysts use computing tools for collaboratively analyzing information in the real world. The research produces a collaborative visualization tool that leverages structured techniques from the intelligence community as well as design knowledge of team awareness from the CSCW (Computer-Supported Collaborative Work) community. The classroom studies evaluate design choices and help identify emerging design challenges. The dissertation ends with design implications and proposes that structured techniques and collaboration be enablers for each other.