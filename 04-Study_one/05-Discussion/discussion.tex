\documentclass[]{article}
\usepackage{lmodern}
\usepackage{amssymb,amsmath}
\usepackage{ifxetex,ifluatex}
\usepackage{fixltx2e} % provides \textsubscript
\ifnum 0\ifxetex 1\fi\ifluatex 1\fi=0 % if pdftex
  \usepackage[T1]{fontenc}
  \usepackage[utf8]{inputenc}
\else % if luatex or xelatex
  \ifxetex
    \usepackage{mathspec}
  \else
    \usepackage{fontspec}
  \fi
  \defaultfontfeatures{Ligatures=TeX,Scale=MatchLowercase}
\fi
% use upquote if available, for straight quotes in verbatim environments
\IfFileExists{upquote.sty}{\usepackage{upquote}}{}
% use microtype if available
\IfFileExists{microtype.sty}{%
\usepackage{microtype}
\UseMicrotypeSet[protrusion]{basicmath} % disable protrusion for tt fonts
}{}
\usepackage[unicode=true]{hyperref}
\hypersetup{
            pdfborder={0 0 0},
            breaklinks=true}
\urlstyle{same}  % don't use monospace font for urls
\usepackage[]{biblatex}
\addbibresource{../../Bibliography/Dissertation.bib}
\IfFileExists{parskip.sty}{%
\usepackage{parskip}
}{% else
\setlength{\parindent}{0pt}
\setlength{\parskip}{6pt plus 2pt minus 1pt}
}
\setlength{\emergencystretch}{3em}  % prevent overfull lines
\providecommand{\tightlist}{%
  \setlength{\itemsep}{0pt}\setlength{\parskip}{0pt}}
\setcounter{secnumdepth}{0}
% Redefines (sub)paragraphs to behave more like sections
\ifx\paragraph\undefined\else
\let\oldparagraph\paragraph
\renewcommand{\paragraph}[1]{\oldparagraph{#1}\mbox{}}
\fi
\ifx\subparagraph\undefined\else
\let\oldsubparagraph\subparagraph
\renewcommand{\subparagraph}[1]{\oldsubparagraph{#1}\mbox{}}
\fi

% set default figure placement to htbp
\makeatletter
\def\fps@figure{htbp}
\makeatother


\date{}

\begin{document}

\section{Discussion}\label{discussion}

\subsection{Reflections on method}\label{reflections-on-method}

The goal of the study is to explore design opportunities to support
collaborative information analysis by evaluating tool usage in a natural
environment over extended period of time. This is to complement research
that only tests tools in short term lab studies (e.g.~e.g.
\autocites{Convertino2012}{Goyal2016}). Due to the constraint of time
(usually about one hour), these studies had to employ a simplified task
with significantly reduced content and complexity. Teams would then not
create complex information artifacts and had less difficulty balancing
limited cognition between problem solving and team coordination. More
complex information artifacts and higher cost of coordination would have
provided more insights into team process of combining information and
tool usage for integrating efforts.

Besides, the limited time may prevent teams from developing sufficient
awareness to work properly. Participants in lab studies face a fresh new
setting: a new formed team, a new collaborative tool, a new task, and a
new environment. It takes time for teams to establish common ground, and
to learn to appropriate the tool to best serve their team. Allowing
teams for more time to explore and make trials would have provided an
opportunity to observe how team awareness have developed and how teams
have appropriated the tool to best solve the problem.

Our classroom study attempts to gain deeper insights on collaborative
information analysis behavior by simulating real world settings in at
least two aspects:

\begin{enumerate}
\def\labelenumi{\arabic{enumi}.}
\item
  The study was one-week long. Teams were able to explore multiple
  strategies to solve the problem, and to change a strategy if they
  encountered a problem. For example, two teams decided to change the
  use of the tool halfway in their analysis. One team started with
  dividing work by case documents, but later decided members should
  annotate different entity types. Another team started with an
  accretion strategy by annotating all entities. Later they discovered
  that this strategy brought too much noise instead of useful
  information, and decided to clean out irrelevant entities (filtering
  strategy). Such change occurs as a consequence of increased awareness
  of team functions and tool capabilities, which takes time to develop.
\item
  Participants in this study are being trained to be professional
  analysts. Before our study they had already been introduced to the
  information analysis techniques and indeed to the state-of-the-art
  tools that support this task. In the study, participants often
  compared CAnalytics to those tools, as well as teamwork with
  CAnalytics to experience with those tools. Therefore their feedback is
  likely to provide deeper insight into strength and weakness of
  CAnalytics. We also purposefully formed three-person teams as opposed
  to dyads because a three-person team would perform more interesting
  social interactions and possible problems, such as cognitive
  specialization \autocite{Borge2014}, social loafing
  \autocite{Karau1993}, and hidden profile \autocite{Stasser2003b}.
\end{enumerate}

Yet classroom study also has limits. For example, many factors and
variables could exist that affect team performance. The fact that these
factors are often impossible to model or control adds to the difficulty
in data analysis (e.g.~identifying performance correlated factors with
linear regression). Also, data collection is challenging because team
interactions are not always accessible. Teams can choose to work
synchronously or asynchronously, and it is difficult to predict when or
where the interaction of most interest is to occur. Due to these limits,
result of classroom study is more likely to identify problems and
generate hypotheses, while lab studies and case studies can be conducted
to evaluate solutions and validate hypotheses with greater control and
deeper access.

\subsection{Reflections on result}\label{reflections-on-result}

A misconception about information analysis is that data modeling and
data analysis are two staged activities. Most analytic tools assume data
has already been modeled and ready to be visualized and analyzed. This
is akin to the waterfall software development model, which features a
sequential process that flows downwards through the phases of
requirement conception, software design, software implementation,
testing and maintenance. Critics have pointed out that the staged
approach may not work properly, because clients may not know exactly
what their requirements are before they see the working software and
designers may not be fully aware of future difficulties in a new
software product. Instead, an iterative design process is often required
that leads to reframe user requirements, redesign, redevelopment, and
retesting.

Similarly, relying only on information that has already been modeled and
delivered to analysts will probably not solve all analytical problems
\autocite{Heuer1999}. It will probably be necessary to look elsewhere,
re-model the data, and dig for more information. As suggested by the
Data-Frame theory, constructing a frame is essential to determine how to
model the data. Contemporary hypotheses help frame the problem, identify
the gap, and determine what data to model. Re-modeling of the data could
lead to a different picture of the problem (e.g.~adding a link between
two clusters changes the layout of the network), leading to completely
different analytic path. The interaction log we captured demonstrated
such an iterative analysis process and the positive subject feedback
confirmed data modeling as an integral part of analysis.

We noted the importance of representing uncertainty. We observed teams
in our study spontaneously employed two different approaches to deal
with team uncertainty given that the tool did not include specific
support: to hold facts and inferences in separate artifacts, and place
facts and inferences of high certainty in an artifact. This demonstrates
both challenge and opportunity to design for uncertainty support. We
propose that a richer graphic language be designed so that analysts can
encode uncertainty into the network view. Links and entities with
different uncertainty are visualized in different colors. Users can
\emph{filter by uncertainty} so that users can choose to have only facts
to take into account or confront all inferences to make a review.

We found that participants tended to make more visible contributions
than valuable contributions, and collaborative technology only made some
activities more visible, thus unintentionally encouraged participants to
do more these activities, although they might not be useful at all. This
is especially true when team practice has not been established and thus
can be easily shaped by outside incentives, such as the awareness
features in our system. We should be cautious to distinguish between
awareness system and contribution system. A contribution system should
only include factors that bear value to the task, such as hypotheses
created and validated in the context of information analysis, and should
be explicitly displayed to users to highlight the value of these
factors. Awareness system, on the other hand, should share all relevant
activities (and perhaps highlight information most relevant to the
current user, for example, when teammates edit your entity). The
information, while valuable to the task goal, can motivate teammates to
contribute in the same direction, and can remind teammates to pull you
back when it deviates from the team goal (e.g.~when one user created too
many low-level entities in Team 108)

A possible design solution to a cluttered display when the number of
entities increases is to enable collapsible data views. Indeed we found
analysts often engaged in multi-level analysis in parallel, frequently
coordinating between, say, confirmation of a location, to associating
sequence of actions, to comparing two groups of evidence, to overviewing
robberies as a whole. A collapsible view can help analysts focus
attention on a certain level of details, and when in collaboration, draw
teammate's attention to the specific item in your intention.

One major critic was the lack of view sharing support in the tool. In
addition to data sharing, we find that views of data should become
shareable resources as well. With the identical data pool, analysts
often have different views of data. For example, analysts can apply a
filter to have a reduced data view, highlight an area to sharpen
analytic focus, and re-layout the node-link graph to cluster relevant
entities. While the data pool represents the information the team have
available, individual views of the data reflect analyst's
\emph{interpretation} toward the information. The views together with
the underlying data embody user's intermediate analytic status.
Therefore we propose that just like data, views, which embodies
interpretation of data, should be shareable.

Views as resources should also be extensible and reusable. For example,
several participants reflected that there were situations when they
found a collaborator's view useful and wanted to build their own work
upon that view without manually reproducing the view. With views as
resources, individuals can take the views to their need. They can also
deliberatively share their own view when they feel other collaborators
will be interested. Shared views are interactive rather than static
images, so that analysts can still perform full functions including
filtering and highlighting, and are able to evolve the view with
collective team efforts, a critical requirement emphasized in
\autocite{Carroll2013}

Our study suggests that the instrument plays an important role in
shaping students' behavior towards more collaborative learning. With
traditional single-user tools, students often employ a
divide-and-conquer strategy; they divide their job responsibility, work
individually on their own part, and put the results together in the end
for report submission. In our study, we observed that students
spontaneously conducted closer collaboration and enjoyed being able to
contribute simultaneously.

Anecdotal reflections from the instructor suggested that the system can
include support for instructor intervention. During the study, the
instructor would go over to students and check their computer screen
about how they were doing, and provide guidance if necessary. The
instructor commented that he valued students' thinking and reasoning
process, and believed that monitoring and guiding students' ongoing
performance would be a valuable supplement to classroom instruction. As
claimed by Heuer \autocite{Heuer1999}, training will be more effective
if supplemented with ongoing advice and assistance. CAnalytics could do
more in supporting the instructor. CAnalytics already provides an
integrated workspace for data modeling, information analysis and
hypothesis generation, and thus makes it easier to monitor the whole
analytic process. Besides, students' interaction logs are already
captured (for team awareness and research purpose only now), and could
be streamed to the instructor for performance monitoring. The process
data provides the instructor a new window to assess students'
performance and to provide intervention when necessary, as suggested by
learning analytics techniques \autocite{Siemens2011}.

\printbibliography

\end{document}
