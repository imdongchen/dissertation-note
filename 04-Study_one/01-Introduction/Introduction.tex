\documentclass[]{article}
\usepackage{lmodern}
\usepackage{amssymb,amsmath}
\usepackage{ifxetex,ifluatex}
\usepackage{fixltx2e} % provides \textsubscript
\ifnum 0\ifxetex 1\fi\ifluatex 1\fi=0 % if pdftex
  \usepackage[T1]{fontenc}
  \usepackage[utf8]{inputenc}
\else % if luatex or xelatex
  \ifxetex
    \usepackage{mathspec}
  \else
    \usepackage{fontspec}
  \fi
  \defaultfontfeatures{Ligatures=TeX,Scale=MatchLowercase}
\fi
% use upquote if available, for straight quotes in verbatim environments
\IfFileExists{upquote.sty}{\usepackage{upquote}}{}
% use microtype if available
\IfFileExists{microtype.sty}{%
\usepackage{microtype}
\UseMicrotypeSet[protrusion]{basicmath} % disable protrusion for tt fonts
}{}
\usepackage[unicode=true]{hyperref}
\hypersetup{
            pdfborder={0 0 0},
            breaklinks=true}
\urlstyle{same}  % don't use monospace font for urls
\usepackage[]{biblatex}
\addbibresource{../../Bibliography/Dissertation.bib}
\IfFileExists{parskip.sty}{%
\usepackage{parskip}
}{% else
\setlength{\parindent}{0pt}
\setlength{\parskip}{6pt plus 2pt minus 1pt}
}
\setlength{\emergencystretch}{3em}  % prevent overfull lines
\providecommand{\tightlist}{%
  \setlength{\itemsep}{0pt}\setlength{\parskip}{0pt}}
\setcounter{secnumdepth}{0}
% Redefines (sub)paragraphs to behave more like sections
\ifx\paragraph\undefined\else
\let\oldparagraph\paragraph
\renewcommand{\paragraph}[1]{\oldparagraph{#1}\mbox{}}
\fi
\ifx\subparagraph\undefined\else
\let\oldsubparagraph\subparagraph
\renewcommand{\subparagraph}[1]{\oldsubparagraph{#1}\mbox{}}
\fi

% set default figure placement to htbp
\makeatletter
\def\fps@figure{htbp}
\makeatother


\date{}

\begin{document}

\section{Introduction}\label{introduction}

Collaborative information analysis is a form of sensemaking wherein a
team analyzes a complex information space of facts and relationships to
identify and evaluate causal hypotheses. A common example of
collaborative information analysis is crime investigation; a variety of
putative facts are assembled, including financial records, witness
observations and interviews, and social connections of various sorts
among persons of interest, from which investigators collaboratively
assess means, motives, and opportunities, articulate and investigate
further hypotheses and deductions, and develop one or more theories of
the crime. Other examples include intelligence analysis, business
intelligence, scientific research, and social constructivist learning.

A critical challenge for information analysts is building an adequate
preliminary data model from textual documents, and insuring that the
data model is employed effectively in hypothesis development and
evaluation. This is an open challenge \autocite{Badalamente2005}.
Standard methods often do not support it at all; for example, Analysis
of Competing Hypotheses (ACH) assumes that data has been modeled, and
that relevant evidence can be adduced appropriately to various
hypotheses, but provides no structured support for either. Other
techniques, such as Information Extraction and Weighting (IEW), help
structure modeling of evidence, but do not extend utilization of
evidence to hypothesis generation. We therefore are motivated to develop
an integrated workspace in which analysts can model and analyze data in
one place, and investigate how that will affect analytic process.

Any work of information analysis at a non-trivial scale is fundamentally
collaborative. A key enabler for effective collaboration is
\emph{activity awareness}, defined as team's awareness of its own
sustained collaborative activity \autocite{Carroll2006}. Derived from
Activity Theory, activity awareness transcends synchronous awareness of
who collaborators are, where a collaborator is looking, etc. It
encompasses issues of many different kinds of information covering all
aspects of an activity, such as events, tasks, goals, mediating
artifacts, social interactions, and group values and norms, which
becomes higher demanding as the activity becomes more complicated.
Awareness support in such a complex activity of information analysis is
perhaps also more challenging than many other situations
(e.g.~collaborative writing) or casual social settings (e.g.
\autocite{Greenberg2001}). Teammates could be working with much more
complex data structure, mediating through multiple analytic artifacts,
and making sense of different levels of analysis, assumptions, and
hypotheses, both synchronously and asynchronously throughout a long-term
course of collaborative interaction. Hence we will investigate how
technology can mediate team collaboration in a complex analytic task
over extended period of time.

We situate our study in classroom learning of information analysis.
Classroom study provides a natural environment in which participants
engage in long term, complex class projects. The students are in fact
learning to be information analysts, and are graded in those courses on
their ability to understand and enact professional practices of
information analysis. This strong normative emphasis on problem solving
practices is a great evaluation context for new interactive tools: Tools
are only valuable to the students insofar as they actually support
better practices and better outcomes.

The study also provides us an opportunity to examine learning experience
with collaborative tools in education of information analytics. Learning
information analytics is challenging because it requires students to not
only get familiar with analytic techniques but also to be able to apply
them in an effective manner. Research has called for enhancement of
information analysis training and education in the context of
globalization characterized by an increasing degree of complexity and
unprecedented acceleration of change, especially the need for
incorporating innovative technology to transform exploding data into
meaningful and actionable information \autocite{Martin2014}.

We thus are motivated to investigate the feasibility, effectiveness and
consequence of supporting integrated data modeling and analysis, as well
as supporting activity awareness in complex information analysis, in the
context of classroom study. We have developed a tool that includes
annotation for data modeling, interactive visualization for data
analysis, and a set of awareness features. For the balance of this
paper, we describe the tool we have developed, the classroom settings,
and our observations. We conclude with design implications derived from
the study as well as future work.

\printbibliography

\end{document}
