\documentclass[]{article}
\usepackage{lmodern}
\usepackage{amssymb,amsmath}
\usepackage{ifxetex,ifluatex}
\usepackage{fixltx2e} % provides \textsubscript
\ifnum 0\ifxetex 1\fi\ifluatex 1\fi=0 % if pdftex
  \usepackage[T1]{fontenc}
  \usepackage[utf8]{inputenc}
\else % if luatex or xelatex
  \ifxetex
    \usepackage{mathspec}
  \else
    \usepackage{fontspec}
  \fi
  \defaultfontfeatures{Ligatures=TeX,Scale=MatchLowercase}
\fi
% use upquote if available, for straight quotes in verbatim environments
\IfFileExists{upquote.sty}{\usepackage{upquote}}{}
% use microtype if available
\IfFileExists{microtype.sty}{%
\usepackage{microtype}
\UseMicrotypeSet[protrusion]{basicmath} % disable protrusion for tt fonts
}{}
\usepackage[unicode=true]{hyperref}
\hypersetup{
            pdfborder={0 0 0},
            breaklinks=true}
\urlstyle{same}  % don't use monospace font for urls
\usepackage[]{biblatex}
\addbibresource{../../Bibliography/Dissertation.bib}
\IfFileExists{parskip.sty}{%
\usepackage{parskip}
}{% else
\setlength{\parindent}{0pt}
\setlength{\parskip}{6pt plus 2pt minus 1pt}
}
\setlength{\emergencystretch}{3em}  % prevent overfull lines
\providecommand{\tightlist}{%
  \setlength{\itemsep}{0pt}\setlength{\parskip}{0pt}}
\setcounter{secnumdepth}{0}
% Redefines (sub)paragraphs to behave more like sections
\ifx\paragraph\undefined\else
\let\oldparagraph\paragraph
\renewcommand{\paragraph}[1]{\oldparagraph{#1}\mbox{}}
\fi
\ifx\subparagraph\undefined\else
\let\oldsubparagraph\subparagraph
\renewcommand{\subparagraph}[1]{\oldsubparagraph{#1}\mbox{}}
\fi

% set default figure placement to htbp
\makeatletter
\def\fps@figure{htbp}
\makeatother


\date{}

\begin{document}

\section{Related work}\label{related-work}

Many researches have been conducted to investigate design features and
tools to support collaborative information analysis. For example, Goyal
and Fussell \autocite{Goyal2016} studied the effect of hypotheses
sharing on sensemaking. Mahyar and Tory \autocite{Mahyar2013} designed a
visualization to connect collaborators' common findings and evaluated
its support for team performance. Hajizadeh et al.
\autocite{Hajizadeh2013} explored how sharing teammate's interactions
affects awareness. These studies report interesting result of controlled
lab studies to validate hypotheses of specific design features. However,
they do not provide insight on how teams would collaborate in the real
world over extended period of time.

Several field studies were conducted aimed to understand design
requirements of collaborative information analysis in more realistic
setting. Chin et al. \autocite{Chin2009} observed and analyzed the
analytic strategies, work practices, tools ad collaboration norms of
professional intelligence analysts. Kang and Stasko \autocite{Kang2011}
studied how student analysts, as in our study, completed in-class
intelligence projects. Carroll et al. \autocite{Carroll2013} attempted
to model a complex analytic task scenario in a lab setting, and examined
the development of team awareness in a four-hour-long task. These
studies helped improve understanding of current work practice with
commercial tools or no supporting tools at all. However, it remains
unknown how teams would behave with technology that is explicitly
designed for collaborative information analysis.

Before the study with our tool, we observed how teams of students
accomplished a course project in the class. Students practiced applying
IEW to extract and model data from documents. They then replicated key
facts into analytic artifacts such as an ACH Matrix in PARC ACH, a
timeline and a network graph in Analyst's Notebook. However, these tools
lack serious collaboration support. Analysts were unable to contribute
simultaneously. The analysis work was often divided by tools: each
individual created and analyzed an artifact with a tool on their own.
This had the consequence that findings and hypotheses be made without
integrating collective efforts and diverse knowledge. Analysts must
coordinate work by manually sharing notebooks or graphs, resulting in a
scattered placement of results, requiring repeated manual
resynchronizing to identify redundant or missing pieces of information,
analysis of information, and analytic hypotheses. The instructor and
students in our study were aware of the shortcomings of available tools
with respect to support of collaboration.

\printbibliography

\end{document}
